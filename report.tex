\documentclass[11pt, a4paper,titlepage]{article}
\usepackage[utf8]{inputenc}
\usepackage[T1]{fontenc}
\usepackage{fixltx2e}
\usepackage{graphicx}
\usepackage{longtable}
\usepackage{float}
\usepackage{wrapfig}
\usepackage{textcomp}
\usepackage{hyperref}
\usepackage[bottom]{footmisc}
\tolerance=1000
\usepackage[left=2.35cm, right=3.35cm, top=3.35cm, bottom=3.35cm]{geometry}
\usepackage[utf8]{inputenc}
\usepackage[greek,english]{babel}
\usepackage{titlesec}
\usepackage{tocbibind}
\makeatletter
\def\@seccntformat#1{%
  \expandafter\ifx\csname c@#1\endcsname\c@section\else
  \csname the#1\endcsname\quad
  \fi}
\makeatother
\begin{document}

\include{title}
\setcounter{tocdepth}{3}

\tableofcontents
\newpage

\section*{Question 1: Protein Family}
\addcontentsline{toc}{section}{Question 1: Protein Family}

\texttt{Q:} Which protein family does the egl-3 encoded protein belong to?


\section*{Question 2: Tissues and Gene Expression}
\addcontentsline{toc}{section}{Question 2: Tissues and Gene Expression} 

\texttt{Q:} In which tissues is egl-3 expressed in C. Elegans?


\section*{Question 3: Development stages}
\addcontentsline{toc}{section}{Question 3: Development stages}

\texttt{Q:} In which developmental stage is the egl-3 expressed?


\section*{Question 4: Mutant strains}
\addcontentsline{toc}{section}{Question 4: Mutant strains}

\texttt{Q:} Are there mutant strains for this gene available? What has
been mutated in the gene? What is the resulting phenotype?


\section*{Question 5: RNA interference phenotype}
\addcontentsline{toc}{section}{Question : RNA interference phenotype5}

\texttt{Q:} What is the RNA-interference phenotype?


\section*{Question 6: RNAi vs Deletion mutant phenotypes}
\addcontentsline{toc}{section}{Question 6: RNAi vs Deletion mutant phenotypes}

\texttt{Q:} Are there differences between RNAi and deletion mutant
phenotypes? If yes, explain why this is possible.


\section*{Question 7: Orthologous proteins}
\addcontentsline{toc}{section}{Question 7: Orthologous proteins}

\texttt{Q:} Are there orthologous proteins in:
\begin{itemize}
\itemsep0em 
\item Other nematodes?
\item Insects?
\item Vertebrate species (non human)?
\item Human?
\item Unicellular organisms?
\end{itemize}
Indicate for each of these species the biological process in which the
protein is involved in and how this function was assessed


\section*{Question 8: Function in orthologous genes}
\addcontentsline{toc}{section}{Question 8: Function in orthologous genes}

\texttt{Q:} Is the function of this orthologous gene in yeast similar
to its function in C. Elegans?


\section*{Question 9: Null mutant in yeast}
\addcontentsline{toc}{section}{Question 9: Null mutant in yeast}

\texttt{Q:} Is a null mutant for this gene available in yeast? What is
the phenotype?


\section*{Question 10: Scientific information availability}
\addcontentsline{toc}{section}{10: Scientific information availability}

\texttt{Q:} What is the latest scientific information regarding the
orthologue of this gene in yeast in relation to cell fusion and the
process of mating in yeast?


\section*{Question 11: Model organism to further study}
\addcontentsline{toc}{section}{11: Model organism to further study}

\texttt{Q:} Which model organism would you choose to further study
this gene and why?


\bibliographystyle{plain} \bibliography{bib-db}
\end{document}
