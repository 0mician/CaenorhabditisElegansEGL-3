\documentclass[11pt, a4paper,titlepage]{article}
\usepackage[utf8]{inputenc}
\usepackage[T1]{fontenc}
\usepackage{fixltx2e}
\usepackage{graphicx}
\usepackage{longtable}
\usepackage{float}
\usepackage{wrapfig}
\usepackage{textcomp}
\usepackage{hyperref}
\usepackage[bottom]{footmisc}
\tolerance=1000
\usepackage[left=2.35cm, right=3.35cm, top=3.35cm, bottom=3.35cm]{geometry}
\usepackage[utf8]{inputenc}
\usepackage[greek,english]{babel}
\usepackage{titlesec}
\usepackage{tocbibind}
\makeatletter
\def\@seccntformat#1{%
  \expandafter\ifx\csname c@#1\endcsname\c@section\else
  \csname the#1\endcsname\quad
  \fi}
\makeatother
\begin{document}

\include{title}
\setcounter{tocdepth}{3}

\tableofcontents
\newpage

\section*{Question 1: Protein family}
\addcontentsline{toc}{section}{Question 1: Protein family}

\texttt{Q:} Which protein family does the egl-3 encoded protein belong to?
\smallskip

\noindent\texttt{A:} The egl-3 gene encodes a homolog of a mammalian
proprotein convertase that participates in peptide secretion. Mature
mRNA is translated to a polypeptide of which not all parts are
important to the mature, activated protein. Parts will be cleaved, and
that is typically done by a proprotein convertase.

This information was found on the Wormbase website, on the egl-3 gene
page. Then wikipedia helped to discover the function of proprotein
convertase (understanding which may help to frame the context of
subsequent questions).

\section*{Question 2: Tissues and gene expression}
\addcontentsline{toc}{section}{Question 2: Tissues and gene expression} 

\texttt{Q:} In which tissues is egl-3 expressed in C. Elegans?
\smallskip

\noindent\texttt{A:} The information can be found in the
\emph{Location} tab of the wormbase website. It is expressed in many
tissues, notably the nervous tissues, in the tail, in the pharynx, and
in the head.

\section*{Question 3: Development stages}
\addcontentsline{toc}{section}{Question 3: Development stages}

\texttt{Q:} In which developmental stage is the egl-3 expressed?
\smallskip

\noindent\texttt{A:} In the same tab, you can see that it is expressed
in postembryonic stages, ie the adult and larvae stages.

\section*{Question 4: Mutant strains}
\addcontentsline{toc}{section}{Question 4: Mutant strains}

\texttt{Q:} Are there mutant strains for this gene available? What has
been mutated in the gene? What is the resulting phenotype?
\smallskip

\noindent\texttt{A:} (additional instruction: take out the million
mutant project). You can find the mutations in the \emph{Phenotype}
section of the website. At the time of writing this report, there were
6. Mutation nr2090, no strain available. Mutations N150 from strain
MT150 (click on the \emph{variation} to see). Mutation N588 from
strain MT1218. Mutation N589 from strain MT1219. Mutation N729 from
strain MT1541. And finally mutation gk238, with strain VC461.

For the phenotype, N150 is a substitution mutation inducing several
different phenotypes. It is \emph{TS} (temperature sensitive
allele). On the page for the allele, you will see the specific
mutation (C->T) in the \emph{Molecular Details} tab.

Other phenotypes include bag of worms, bloated, coiler, egg laying
defective, egg retention, egg laying variant, post translational
processing variant. You can get more info on what those phenotypes
mean by clicking on them. You may be asked if another MO has a similar
phenotype, but the names may be different, so understanding may help
answering those kind of questions. If different MO have the same kind
of phenotypes (like stop eating properly), you may infer that the gene
may be functionally well conserved. You may also know of partner
interactions in one MO and search for those in the other MO.

\section*{Question 5: RNA interference phenotypes}
\addcontentsline{toc}{section}{Question 5: RNA interference phenotypes}

\texttt{Q:} What is the RNA-interference phenotype?
\smallskip

\noindent\texttt{A:} In the same location, on the gene page,
\emph{phenotype} tab. There are a couple of additional alternative
phenotypes observed with RNAi, including body wall muscle sarcomere
morphology variant, extended life span, mitochondria morphology
variant muscle, and protein degradation variant.

\section*{Question 6: RNAi vs deletion mutant phenotypes}
\addcontentsline{toc}{section}{Question 6: RNAi vs deletion mutant phenotypes}

\texttt{Q:} Are there differences between RNAi and deletion mutant
phenotypes? If yes, explain why this is possible.
\smallskip

\noindent\texttt{A:} this question is a bit tricky, you have to rely
on tested evidences to confirm (assays). But, some of the RNAi have no
\emph{Allele} as supporting evidences, meaning that they have not been
tested (for example, the \emph{body wall muscle sarcomere morphology
  variant} which has only RNAi as supporting evidence.

Should you expect a difference? There could be differences in PTM, and
if the RNAi is used, the protein never reaches the stage of
translation. If you have a deletion mutant, you may also observe a
difference because RNAi (knock-down) is not as efficient as a
knock-out of the gene. Many functions are dose-dependent, so if you
have nothing, the function would be off, but with a little bit, you
may still have partial function. Also, getting RNAi may be difficult
to insert in some tissues (esp. nervous tissues).

\section*{Question 7: Orthologous proteins}
\addcontentsline{toc}{section}{Question 7: Orthologous proteins}

\texttt{Q:} Are there orthologous proteins in:
\begin{itemize}
\itemsep0em 
\item Other nematodes?
\item Insects?
\item Vertebrate species (non human)?
\item Human?
\item Unicellular organisms?
\end{itemize}
Indicate for each of these species the biological process in which the
protein is involved in and how this function was assessed
\smallskip

\noindent\texttt{A:} the go-to database to answer this question is
\emph{NCBI homologene}. You will directly see many results there
because the gene is well conserved. You could also check the Homology
tab of the gene page on wormbase. For other nematodes, that may be the
way to go since nematodes are wormbase's prime focus. Often, it is ok
to start with NCBI to switch and investigate multiple organisms, but
after first review, you should then go to the organism's db for up to
date information (as they are updated more frequently).

\begin{itemize}
\itemsep0em 
\item Other nematodes: there are quite a few listed on Wormbase
  (C. remanei, C. brenneri, C. briggsae, C. japonica,
  P. pacificus). egl-3 is very conserved in nematodes, all related to
  peptidase activity (infered from orthology)
\item Insects: there is a verified homolog in drosophilia (Flybase:
  amon gene). This protein has a peptidase activity (inferred from
  direct assay).
\item Vertebrate species (non human): for example D. Rerio (ZFIN
  database, gene name is pcsk2, molecular function). Hydrolase and
  peptidase activities infered from experimental assays.
\item Human: same gene (pcsk2).
\item Unicellular organisms: no good homolog gene in Yeast. Why is
  that? Because neuropeptide are communication molecules between
  nervous system components. Yeast does communicate to, but they do
  not have neuropeptide. The only distant gene homolog (KEX2) is
  related to the same family as egl-3. These proteins cut out peptides
  to activate them, which is a very general function, and that's why
  there is a homolog. In higher organism these branched out to
  specific uses (some cut neuropeptides for example). The function of
  KEX2 is to encode a a2+ dependent serine protease involved in
  proprotein processing, infered from direct assay. (SGD database,
  KEX2 gene page, gene ontology, molecular function).
\end{itemize}

\section*{Question 8: Function in orthologous genes}
\addcontentsline{toc}{section}{Question 8: Function in orthologous genes}

\texttt{Q:} Is the function of this orthologous gene in yeast similar
to its function in C. Elegans?
\smallskip

\noindent\texttt{A:} To find the function, you can search the SGD
website, on the KEX2 page, it is a calcium-dependent serine protease
involved in the activation of proproteins of the secretory
pathway. Neuropeptides are also secreted proteins, but this KEX2 is
not as specific (it is involved in a diverse array of secreted
proteins - see SGD, summary paragraph).

\section*{Question 9: Null mutant in yeast}
\addcontentsline{toc}{section}{Question 9: Null mutant in yeast}

\texttt{Q:} Is a null mutant for this gene available in yeast? What is
the phenotype?
\smallskip

\noindent\texttt{A:} SGD, phenotype tab. There are a couple of mutants
available there. With resistance decreased to alkaline pH, to acid pH,
and other phenotypes suc as cell size increased, competitive fitness
decreased, ...

\section*{Question 10: Scientific information availability}
\addcontentsline{toc}{section}{Question 10: Scientific information availability}

\texttt{Q:} What is the latest scientific information regarding the
orthologue of this gene in yeast in relation to cell fusion and the
process of mating in yeast?
\smallskip

\noindent\texttt{A:} SGD, literature tab. Search for keywords: Cell
fusion, and mating. Reference: Aguilar PS, et al. (2010) Structure of
sterol aliphatic chains affects yeast cell shape and cell fusion
during mating. Proc Natl Acad Sci U S A 107(9):4170-5. Or another
paper is: Heiman MG, et al. (2007) The Golgi-resident protease Kex2
acts in conjunction with Prm1 to facilitate cell fusion during yeast
mating. J Cell Biol 176(2):209-22 PMID:17210951 (which is the paper
this question was refering to)

\section*{Question 11: Model organism to further study}
\addcontentsline{toc}{section}{Question 11: Model organism to further study}

\texttt{Q:} Which model organism would you choose to further study
this gene and why?
\smallskip

\noindent\texttt{A:} this question was deemed too vague, because it
actually depends on what you want to study. Some MO seem odd anyway
regardless of the question, like yeast.

\bibliographystyle{plain} \bibliography{bib-db}
\end{document}
